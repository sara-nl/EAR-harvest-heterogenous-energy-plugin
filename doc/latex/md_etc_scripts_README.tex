A set of bash scripts are provided to make some processes as simple and contained as possible. The two main goals of these scripts are the following\+: 1) Automatize the learning phase process. 2) If you are not using {\itshape S\+L\+U\+RM}\+:
\begin{DoxyItemize}
\item Launch the E\+AR daemon through {\itshape S\+SH}.
\item Launch jobs by M\+PI commands. In this case, the scripts are responsible of launching your applications together with E\+AR library.
\end{DoxyItemize}

\subsection*{Subfolders and contents }

The directory {\ttfamily etc} placed inside the installation path contains a subfolder called {\ttfamily scripts}. Inside, you will find the following subfolders\+:

The {\ttfamily learning} folder, which contains a set of scripts to compile, test and run the learning phase.
\begin{DoxyItemize}
\item {\bfseries learning\+\_\+phase\+\_\+compile.\+sh}\+: compiles in the local node all the benchmarks that are part of the learning phase.
\item {\bfseries learning\+\_\+phase\+\_\+execute.\+sh}\+: performs the complete tour of the learning phase, including all the benchmarks for all the selected frequencies. This script is launched by your {\ttfamily mpirun} command or the {\ttfamily mpi\+\_\+exec.\+sh} script, not directly executed locally by {\ttfamily ./learning\+\_\+phase\+\_\+execute.sh}. This is because all the nodes have to execute this script completely in isolation, hence first you have to give instructions to those nodes to execute the script.
\item {\bfseries learning\+\_\+phase\+\_\+helper.\+sh}\+: wraps previous learning phase scripts functions, avoiding the repetition of code.
\end{DoxyItemize}

The {\ttfamily running} folder, which contains a set of scripts for launch M\+PI applications. You can launch these scripts with empty parameters to view its usage.
\begin{DoxyItemize}
\item {\bfseries mpi\+\_\+exec.\+sh}\+: contains the {\ttfamily mpirun} command with a prepared environment to launch the application next to the E\+AR library in a set of specified nodes. Depending on the M\+PI distribution, may be is necessary to edit the launching command, because this could make not working other scripts\+: 
\begin{DoxyCode}
1 ## Starting the application
2 mpiexec.hydra -l -genv LD\_PRELOAD=$\{EAR\_LIB\_PATH\} -genvall $\{MPI\_HOST\} -n $\{MPI\} -ppn=$\{PPN\} $\{BINARY\}
\end{DoxyCode}
 So, head to this line and replace the {\ttfamily mpiexec.\+hydra} by {\ttfamily mpirun} similar.
\item {\bfseries ssh\+\_\+daemon\+\_\+start.\+sh}\+: loads the E\+AR daemon in a a specified set of nodes.
\item {\bfseries ssh\+\_\+daemon\+\_\+exit.\+sh}\+: closes the E\+AR daemon of a specified set of nodes.
\end{DoxyItemize}

The {\ttfamily environment} folder, which contains two script which defines the environment of the daemon and the library\+:
\begin{DoxyItemize}
\item {\bfseries ear\+\_\+vars.\+sh}\+: defines the E\+AR environment and therefore the behaviour of the library and daemon by default. Please, head https\+://github.com/\+Barcelona\+Supercomputing\+Center/\+E\+A\+R/blob/development/etc/\+R\+E\+A\+D\+M\+E.\+md \char`\"{}environment variables configuration page\char`\"{} to learn more about this variables and customize this file values to fit the needs of your cluster.
\item {\bfseries lib\+\_\+vars.\+sh}\+: defines third party libraries paths. These paths were written during the {\ttfamily configure}.
\end{DoxyItemize}

The {\ttfamily examples} folder just contains some examples of the use of these scripts.

\subsection*{Script dependancies }

The following picture shows the dependancies between these scripts. This way of organizing favors the simplicity of editing, allowing that smallest change to take effect on all scripts.

 