E\+AR offers the following commands\+:
\begin{DoxyItemize}
\item Commands to analyze data stored in the DB\+: eacct and ereport
\item Commands to control and temporally modify cluster settings\+: econtrol
\item Commands to create/update the DB\+: ear\+\_\+create\+\_\+database
\end{DoxyItemize}

All these commands read the E\+AR configurarion file (ear.\+conf) to determine if the user is an authorized (or not user). Root is a special case, it doesn\textquotesingle{}t need to be included in the list of authorized users. Some options are disables when the user is not authorized.

\section*{Energy Account (eacct)}

The eacct command shows accounting information stored in the E\+AR DB for jobs (and step) I\+Ds. It provides the following options. 
\begin{DoxyCode}
1 Usage: eacct [Optional parameters]
2     Optional parameters: 
3         -h  displays this information
4         -u  specifies the user whose applications will be retrieved. Only available to privileged users.
       [default for authorized users: all users]
5         -j  specifies the job id and step id to retrieve with the format [jobid.stepid]. A user can only
       retrieve its own jobs unless said user is privileged. [default: all jobs]
6         -l  shows the information for each node for each job instead of the global statistics for said job.
7         -n  specifies the number of jobs to be shown, starting from the most recent one. [default: all
       jobs]
8         -t  specifies the energy tag to filter the retrieved jobs. [default: all tags]
9         -c  specifies the file to save the output information in csv format.
\end{DoxyCode}


\section*{Energy report (ereport)}

The ereport command creates reports from the information stored in the E\+AR DB concerning energy. ... Usage\+: ereport \mbox{[}Optional parameters\mbox{]} Optional parameters\+: -\/s start\+\_\+time indicates the starting period from which the energy information will be computed. Format\+: Y\+Y\+Y\+Y-\/\+M\+M-\/\+DD. Default\+: 1970-\/01-\/01. -\/e end\+\_\+time indicates the end of the period from which the energy information will be computed. Format\+: Y\+Y\+Y\+Y-\/\+M\+M-\/\+DD. Default\+: current time. -\/n node\+\_\+name indicates from which node the energy information will be computed. Default\+: none (all nodes computed together). \textquotesingle{}all\textquotesingle{} option will show all nodes computed individually. -\/u user\+\_\+name requests the energy consumed by a user in the selected period of time. Default\+: none (not filtering by user). \textquotesingle{}all\textquotesingle{} option will show all users individually. -\/t \hyperlink{structenergy__tag}{energy\+\_\+tag} requests the energy consumed filtering by the selected energy tag. Default\+: none (not filtering by \hyperlink{structenergy__tag}{energy\+\_\+tag}). \textquotesingle{}all\textquotesingle{} option will show all tags individually. -\/h displays this information

\section*{Energy control (econtrol)}

The econtrol command modifies cluster settings (temporally) related to power policy settings. These options are sent to all the nodes in the cluster.


\begin{DoxyCode}
1 Usage: econtrol [options]
2     --set-freq  newfreq         ->sets the frequency of all nodes to the requested one
3     --set-def-freq  newfreq     ->sets the default frequency
4     --set-max-freq  newfreq     ->sets the maximum frequency
5     --inc-th    new\_th          ->increases the threshold for all nodes
6     --red-def-freq  reduction   ->reduces the default frequency
7     --restore-conf              ->restores the configuration to all nodes
8     --ping                      ->pings all nodes. Additionally, --ping=node\_name pings node\_name
       individually
\end{DoxyCode}


\section*{Database commands}


\begin{DoxyItemize}
\item ear\+\_\+create\+\_\+database\+: Creates the E\+AR DB used for accounting and for the global energy control. Requires root access to the My\+S\+QL server. It reads the ear.\+conf to get connection details and DB name. 
\end{DoxyItemize}