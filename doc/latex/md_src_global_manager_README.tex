The E\+AR Global Manager Daemon (E\+A\+R\+G\+MD) is a cluster wide component that controls the percetage of the maximum energy consumed. It can be configured to take actions automatically like warning sysadmins to take actions or limiting the nodes policy.

\subsection*{Requirements }

E\+A\+R\+G\+MD uses periodic power metrics reported by ../daemon/\+R\+E\+A\+D\+ME.md \char`\"{}\+E\+A\+R\+D\char`\"{}, the per-\/node daemon, including job identification details (job id and step id if you are using the S\+L\+U\+RM plugin). These metrics are stored and aggregated in a Maria\+DB (My\+S\+QL) database through the ../database\+\_\+cache/\+R\+E\+A\+ME.md \char`\"{}\+E\+A\+R\+D\+B\+D\char`\"{}.

\subsection*{Configuration }

The E\+AR Global Manager Daemon uses the {\ttfamily /ear.conf} file to be configured. It can be dynamically configured by reloading the service.


\begin{DoxyCode}
1 # Fields related to the Global Manager Daemon
2 
3 # Verbose level
4 GlobalManagerVerbose=1
5 # Period T1 in seconds (10 min)
6 GlobalManagerPeriodT1=600
7 # Period T2 in seconds (30 days)
8 GlobalManagerPeriodT2=2592000
9 # Abosolute value (joules)
10 GlobalManagerEnergyLimit=756000
11 GlobalManagerPort=6000
12 # Two modes are supported (0 pasive, 1 active or automatic)
13 GlobalManagerMode=0
14 # Anyway, a mail can be sent reporting the warning level (and the action taken in automatic mode). nomail
       means no mail is going to be sent
15 GlobalManagerMail=nomail
16 #Thee values must be provided corresponding with DEFCON\_L4,DEFCON\_L3, and DEFCON\_L2 (higer values means
       PANIC)
17 GlobalManagerWarningsPerc=85,90,95
18 #T1 "grace" periods between DEFCON re-evaluations. In that case, 3 T1 periods means 30 minutes
19 GlobalManagerGracePeriods=3
\end{DoxyCode}


\subsection*{A\+PI }

The global manager offers a simple A\+PI {\ttfamily eargm\+\_\+api.\+c} to notify the execution and finalization of the jobs. The A\+PI just notifies the E\+A\+R\+G\+MD about the number of nodes to be used and released after the execution of the job. The S\+L\+U\+RM plugin automatically does this task.

\subsection*{Execution }

To execute this component, this {\ttfamily systemctl} command examples are provided\+:
\begin{DoxyItemize}
\item {\ttfamily sudo systemctl start eard} to start the E\+A\+R\+G\+MD service.
\item {\ttfamily sudo systemctl stop eard} to stop the E\+A\+R\+G\+MD service.
\item {\ttfamily sudo systemctl reload eard} to force to reload the configuration of the E\+A\+R\+G\+MD service.
\end{DoxyItemize}

\subsection*{Commands }

This is a list of the available commands\+:

\tabulinesep=1mm
\begin{longtabu} spread 0pt [c]{*2{|X[-1]}|}
\hline
\rowcolor{\tableheadbgcolor}{\bf Command }&{\bf Description  }\\\cline{1-2}
\endfirsthead
\hline
\endfoot
\hline
\rowcolor{\tableheadbgcolor}{\bf Command }&{\bf Description  }\\\cline{1-2}
\endhead
eargm\+\_\+new\+\_\+job &Informs the E\+A\+R\+G\+MD that a job is about to start. \\\cline{1-2}
eargm\+\_\+end\+\_\+job &Informs the E\+A\+R\+G\+MD that a job has finished. \\\cline{1-2}
\end{longtabu}
\subsection*{License }

All the files in the E\+AR framework are under the L\+G\+P\+Lv2.\+1 license. See the \href{../../COPYING}{\tt C\+O\+P\+Y\+I\+NG} file in the E\+AR root directory. 