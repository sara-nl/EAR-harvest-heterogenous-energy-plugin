The node daemon is the component in charge of providing any kind of services that requires privileged capabilities. Current version is conceived as an external process executed with root privileges.

The E\+A\+RD provides two basic services, each one covered by one thread\+:
\begin{DoxyItemize}
\item Provides privileged metrics such as average frequency, uncore integrated memory controller counters to compute the memory bandwidth, and also energy metrics (DC node, D\+R\+AM and package energy).
\item Implements a periodic power monitoring service. This service allows E\+AR package to control the total energy consumed in the system.
\end{DoxyItemize}

\subsection*{Requirements }

E\+A\+RD uses C\+P\+U\+Power, Free\+I\+P\+MI and P\+A\+PI (with R\+A\+PL component enabled). Paths to these libraries must be specified during the installation process when they are not installed in default paths.

When executed in production environments, E\+A\+RD connects with ../database\+\_\+cache/\+R\+E\+A\+ME.md \char`\"{}\+E\+A\+R\+D\+B\+D\char`\"{} service, that has to be up before starting the node daemon, otherwise values reported by E\+A\+RD to be stored in the database, will be lost.

\subsection*{Configuration }

The E\+AR Daemon uses the {\ttfamily /ear.conf} file to be configured. It can be dynamically configured by reloading the service.


\begin{DoxyCode}
1 # Fields related to the Daemon
2 # Lines starting with # are comments
3 
4 # Verbose level
5 NodeDaemonVerbose=1
6 
7 # Frequency used by power monitoring service (in seconds)
8 NodeDaemonPowermonFreq=10
9 
10 # Maximum p\_state supported (1 nominal)
11 NodeDaemonMaxPstate=1
12 
13 # Turbo frequencies supported (0 no)
14 NodeDaemonTurbo=0
15 
16 # Port to be used by the remote API
17 NodeDaemonPort=5000
18 
19 # MySQL DB is used
20 NodeUseDB=1
21 
22 # MySQL is used and records are sent using eardbd, otherwise EARD will send it directly to the MySQL server
23 NodeUseEARDBD=1
\end{DoxyCode}


\subsection*{Execution }

To execute this component, this {\ttfamily systemctl} command examples are provided\+:
\begin{DoxyItemize}
\item {\ttfamily sudo systemctl start eard} to start the E\+A\+RD service.
\item {\ttfamily sudo systemctl stop eard} to stop the E\+A\+RD service.
\item {\ttfamily sudo systemctl reload eard} to force to reload the configuration of the E\+A\+RD service.
\end{DoxyItemize}

Log messages are generated during the execution. Use journalctl command to see eard message\+:


\begin{DoxyItemize}
\item {\ttfamily sudo journalctl -\/u eard -\/f} to force to reload the configuration of the E\+A\+RD service.
\end{DoxyItemize}

\subsection*{Reconfiguration }

After executing a \char`\"{}systemctl reload eard\char`\"{} command, not all the E\+A\+RD options are dynamically updated. The list of updated variables are\+:


\begin{DoxyItemize}
\item Node\+Daemon\+Max\+Pstate
\item Node\+Daemon\+Verbose
\item Default policy and default policy settings
\item Node\+Daemon\+Powermon\+Freq
\end{DoxyItemize}

To reconfigure other detauils such as E\+A\+RD port, coefficients, etc, E\+A\+RD must be stopped and restarted again.

\subsection*{A\+PI }

The (node) Daemon offers a simple A\+PI to request changes on the frequency, modify the current node settings, and reload the system configuration by reading {\ttfamily /ear.conf}

Three A\+P\+Is are provided\+:
\begin{DoxyItemize}
\item Local A\+PI, to be used by ../library/\+R\+E\+A\+D\+ME.md \char`\"{}\+E\+A\+R\+L\char`\"{} (or any other runtime). It can be found in \href{eard_api.h}{\tt eard\+\_\+api.\+h}. This A\+PI involves complex data types and is not public.
\item Local A\+PI, to be used by applications. It is a subset of the E\+A\+RD api and designed to be used by any applications to contact the privileged metric service offered by E\+A\+RD. This A\+PI is public and can be used without restrictinos, so doesn\textquotesingle{}t include functions to change the frequency. It can be found at \href{.}{\tt T\+BD}.
\item Remote A\+PI, to be used by the ../global\+\_\+manager/\+R\+E\+A\+D\+ME.md \char`\"{}\+E\+A\+R\+G\+M\+D\char`\"{} or system commands and tools such as the {\ttfamily econtrol}. Can be found at \href{eard_rapi.h}{\tt eard\+\_\+rapi.\+h} and is not public.
\end{DoxyItemize}

\subsection*{Basic commands }

E\+A\+RD includes some basic commands to be used in the same node eard is running. These commands change/get specific settings in the current node. They are not designaed to be used as system tools, just under development systems. We are offering\+:


\begin{DoxyEnumerate}
\item Commands using the eard local api\+:
\begin{DoxyItemize}
\item commands/ear\+\_\+get\+\_\+node\+\_\+freq\+: Usage\+: ./ear\+\_\+get\+\_\+node\+\_\+freq num\+\_\+cpus
\item commands/ear\+\_\+set\+\_\+ondemand\+: Usage\+:ear\+\_\+set\+\_\+ondemand cpus$\vert$\mbox{[}-\/h\mbox{]}. Changes the current governor to ondemand
\item commands/ear\+\_\+set\+\_\+node\+\_\+freq\+: Usage\+: ear\+\_\+set\+\_\+node\+\_\+freq node\+\_\+freq (in M\+Hz) num\+\_\+cpus
\end{DoxyItemize}
\item Commands using the remote api (can be connected from a different node) to notify new job/end job. They do the same functionality than E\+AR slurm plugin does
\begin{DoxyItemize}
\item commands/ear\+\_\+new\+\_\+job\+:Connects with eard and sends a new\+\_\+job requests. It automatically reads ear.\+conf file and sets application details based on environment variables.\+Only needed when slurm is not installed.
\item commands/ear\+\_\+end\+\_\+job\+: Connects with eard and sends an end\+\_\+job request. It automatically reads ear.\+conf file and sets application details based on environmen t variables. Only needed when slurm is not installed.
\end{DoxyItemize}
\item Commands using the remote api (can be connected from a different node) to change node settings (defaul/max frequencies and th (for min\+\_\+time))
\begin{DoxyItemize}
\item commands/ear\+\_\+set\+\_\+freq\+: usage\+:ear\+\_\+set\+\_\+freq new\+\_\+freq (in K\+Hz)
\item commands/ear\+\_\+inc\+\_\+th\+: usage\+:./ear\+\_\+inc\+\_\+th perc to be added to current Performance\+Efficiency\+Gain (5 means 5\%)
\item commands/ear\+\_\+red\+\_\+freq\+: usage\+:./ear\+\_\+red\+\_\+freq p\+\_\+states
\item commands/ear\+\_\+set\+\_\+th\+: usage\+:./ear\+\_\+set\+\_\+th Min\+Performance\+Efficiency\+Gain (75 means 0.\+75\%)
\end{DoxyItemize}
\end{DoxyEnumerate}

\subsection*{Daemon testing }

Look into the ../tests/\+R\+E\+A\+D\+ME.md \char`\"{}\+E\+A\+R test folder\char`\"{}

\subsection*{License }

All the files in the E\+AR framework are under the L\+G\+P\+Lv2.\+1 license. See the \href{../../COPYING}{\tt C\+O\+P\+Y\+I\+NG} file in the E\+AR root directory. 