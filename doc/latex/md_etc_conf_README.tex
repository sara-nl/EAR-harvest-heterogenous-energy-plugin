{\ttfamily ear.\+conf} is a text file describing the E\+AR package behavior in the cluster. It must be readable by all compute nodes and by nodes where commands are executed.

Usually the first word in the configuration file expresses the component related with the option. Lines starting with \# are comments. Not all the options and arguments are required because some of the components or functionallity could be disabled.

A test for ear.\+conf file can be found in the path {\ttfamily src/test/functionals/ear\+\_\+conf}

\section*{Services configuration parameters}

1) Maria\+DB configuration


\begin{DoxyCode}
1 # The IP or hostname of the node where the MariaDB (MySQL) process is running.
2 MariaDBIp=172.30.2.101
3 #
4 MariaDBUser=ear\_daemon
5 #
6 MariaDBPort=0
7 #
8 MariaDBPassw=
9 #
10 MariaDBDatabase=EAR\_DB
\end{DoxyCode}


2) E\+AR Global Manager (E\+A\+R\+G\+MD)


\begin{DoxyCode}
1 # Verbose level
2 GlobalManagerVerbose=1
3 # Period T1 and period T2 are specified in seconds. T1 must be less than T2. (i.e. 10 min and 1 month)
4 GlobalManagerPeriodT1=90
5 GlobalManagerPeriodT2=259200
6 # Units field, Can be '-' (Joules), 'K' KiloJoules or 'M' MegaJoules
7 GlobalManagerUnits=K
8 #
9 GlobalManagerEnergyLimit=550000
10 # Use '1' or not '0' aggregated metrics to compute total energy.
11 GlobalManagerUseAggregated=1
12 # The IP or hostname of the node where the EARGMD demon is running.
13 GlobalManagerHost=hostname
14 # Port where EARGMD will be listening.
15 GlobalManagerPort=50000
16 # Global manager modes. Two modes are supported '0' (manual) or '1' (automatic)
17 GlobalManagerMode=0
18 # A mail can be sent reporting the warning level (and the action taken in automatic mode). 'nomail' means
       no mail is sent. This option is independendent of the node
19 GlobalManagerMail=nomail
20 # Thee values must be provided corresponding with DEFCON\_L4,DEFCON\_L3 and DEFCON\_L2 (higer values means
       PANIC).
21 GlobalManagerWarningsPerc=85,90,95
22 # Number of "grace" T1 periods before doing a new re-evaluation
23 GlobalManagerGracePeriods=3
\end{DoxyCode}


3) E\+AR Daemon (E\+A\+RD)


\begin{DoxyCode}
1 # The verbosity level [0..4]
2 NodeDaemonVerbose=1
3 # Frequency used by power monitoring service, in seconds.
4 NodeDaemonPowermonFreq=60
5 # Maximum supported frequency (1 means nominal, no turbo).
6 NodeDaemonMaxPstate=1
7 # Enable (1) or disable (0) the turbo frequency
8 NodeDaemonTurbo=0
9 # The port where the EARD will be listening
10 NodeDaemonPort=5000
11 # Enables the use of the database
12 NodeUseDB=1
13 # Inserts data to MySQL sendint that data to the EARDBD (1) or directly (0).
14 NodeUseEARDBD=1
15 # '1' means EAR is totally controlling frequencies (targeting production systems) and 0 means EAR will not
       change the frequencies when users are not using EAR library (targeted to benchmarking systems)
16 NodeForceFrequencies=1
\end{DoxyCode}


4) E\+AR Database Manager Daemon (E\+A\+R\+D\+BD)


\begin{DoxyCode}
1 # In seconds, time of accumulating data in every aggregation
2 DBDaemonAggregationTime=60
3 # In seconds, time between insert the buffered data
4 DBDaemonInsertionTime=30
5 # Port where the EARDBD server is listening
6 DBDaemonPortTCP=4711 , This port is used for main EARDBD
7 # Port where the EARDBD mirror is listening
8 DBDaemonPortSecTCP=4712
9 # Port is used to synchronize the server and mirror
10 DBDaemonSyncPort=4713
11 # Memory allocated per process. It means that if there is a server and mirror in a node a double of that
       value will be allocated. It is expressed in MegaBytes.
12 DBDaemonMemorySize=120
13 # The percentage of the memory block used by each type. These types are: mpi, non-mpi and learning
       applications, loops, energy metrics and aggregations and events, in that order.
14 DBDaemonMemorySizePerType=40,20,5,24,5,1,5
\end{DoxyCode}


5) E\+AR Library (E\+A\+RL)


\begin{DoxyCode}
1 # Path where coefficients are installed
2 CoefficientsDir=etc/ear/coeffs
3 # Number of levels used by DynAIS algorithm.
4 DynAISLevels=4
5 # Windows size used by DynAIS, the higher the size the higer the overhead.
6 DynAISWindowSize=500
7 # Maximum time in seconds that EAR will wait until a signature is computed. After this value, if no
       signature is computed, EAR will go to periodic mode.
8 DynaisTimeout=30
9 # Time in seconds to compute every application signature when the EAR goes to periodic mode.
10 LibraryPeriod=30
11 # Number of MPI calls whether EAR must go to periodic mode or not.
12 CheckEARModeEvery=1000
\end{DoxyCode}


6) Paths


\begin{DoxyCode}
1 # Path used for communitation files, shared memory, etc. It must be PRIVATE per compute node and with
       read/write permissions.
2 TmpDir=/tmp/ear
3 # Path where coefficients and configuration are stored. It must be readable in all compute nodes.
4 EtcDir=/etc/ear
5 # Path where metrics are generated in text files when no database is installed. A suffix is included.
6 DataBasePathName=/etc/ear/dbs/dbs.
\end{DoxyCode}


7) Energy policies


\begin{DoxyCode}
1 # Default policy (MONITORING\_ONLY, MIN\_TIME\_TO\_SOLUTION, and MIN\_ENERGY\_TO\_SOLUTION).
2 DefaultPowerPolicy=MIN\_TIME\_TO\_SOLUTION
3 # List of allowed policies for normal users (it is a subset of all three policies).
4 SupportedPolicies=MONITORING\_ONLY,MIN\_TIME\_TO\_SOLUTION
5 # Starting P\_STATE for each policy, specified in the following order: MIN\_ENERGY, MIN\_TIME and
       MONITORING\_ONLY.
6 DefaultPstates=1,4,4
7 # Threshold used for MIN\_TIME\_TO\_SOLUTION policy.
8 MinEfficiencyGain=0.7
9 # Threshold used for MIN\_ENERGY\_TO\_SOLUTION policy.
10 MaxPerformancePenalty=0.1
11 # Minimum time between two energy readings for performance accuracy
12 MinTimePerformanceAccuracy=10000000
\end{DoxyCode}


8) Security

Authorized users that are allowed to change policies, thresholds and frequencies are supposed to be administrators. A list of users, Linux groups, and/or S\+L\+U\+RM accounts can be provided to allow normal users to perform that actions.


\begin{DoxyCode}
1 AuthorizedUsers=user1,user2
2 AuthorizedAccounts=acc1,acc2,acc3
3 AuthorizedGroups=xx,yy
\end{DoxyCode}


Energy tags are pre-\/defined configurations for some applications (E\+AR library is not loaded). This energy tags accept a user ids, groups and S\+L\+U\+RM accounts of users allowed to use that tag.


\begin{DoxyCode}
1 EnergyTag=memory-intensive pstate=4 users=user1,user2 groups=group1,group2 accounts=acc1,acc2
\end{DoxyCode}


9) Special nodes

Describes nodes with some special characteristic such as different default P\+\_\+\+S\+T\+A\+T\+Es, default coefficients file and/or policy thresholds.


\begin{DoxyCode}
1 # The 'MaxPerformancePenalty' and 'MinEfficiencyGain' accepts values between 0 and 1.
2 NodeName=nodename\_list CPUs=24 DefaultPstates=2,5,5   DefCoefficientsSFile=filename
       MaxPerformancePenalty=def\_th MinEfficiencyGain=def\_th
\end{DoxyCode}


10) Island description

Nodes are grouped in islands, this section is mandatory since it is used for cluster description. Normally nodes are grouped in islands shares the same hardware characteristics and also its database daemons (E\+A\+R\+D\+B\+DS).

Remember that there are two kinds of database daemons. One called \textquotesingle{}server\textquotesingle{} and other one called \textquotesingle{}mirror\textquotesingle{}. Both performs the metrics buffering process, but just one performs the insert. The mirror will do that insert in case the \textquotesingle{}server\textquotesingle{} process crashes or the node fails.

It is recommended that all islands to have symmetry. For example, if the island I0 and I1 have the server N0 and the mirror N1, the next island would have to point the same N0 and N1 or point to new ones N2 and N3.

Multiple E\+A\+R\+D\+B\+Ds are supported in the same island, so more than one line per island is required, but the condition of symmetry have to be met.

It is recommended that for a island to the server and the mirror running in different nodes. However, the E\+A\+R\+D\+BD program could be both server and mirror at the same time. This means that the islands I0 and I1 could have the N0 server and the N2 mirror, and the islands I2 and I3 the N2 server and N0 mirror, fullfing the symmetry requirements.


\begin{DoxyCode}
1 Island=0 Nodes=nodename\_list DBIP=EARDB\_server\_hostname   DBSECIP=EARDB\_mirror\_hostname
\end{DoxyCode}


Detailed island accepted values\+:
\begin{DoxyItemize}
\item nodename\+\_\+list accepts the following formats\+:
\begin{DoxyItemize}
\item Nodes={\ttfamily node1,node2,node3}
\item Nodes={\ttfamily node\mbox{[}1-\/3\mbox{]}}
\item Nodes={\ttfamily node\mbox{[}1,2,3\mbox{]}}
\end{DoxyItemize}
\item Any combination of the two latter options will work, but if nodes have to be specified individually (the first format) as of now they have to be specified in their own line. As an example\+:
\begin{DoxyItemize}
\item Valid formats\+:
\begin{DoxyItemize}
\item Island=1 Nodes={\ttfamily node1,node2,node3}
\item Island=1 Nodes={\ttfamily node\mbox{[}1-\/3\mbox{]},node\mbox{[}4,5\mbox{]}}
\end{DoxyItemize}
\item Invalid formats\+:
\begin{DoxyItemize}
\item Island=1 Nodes={\ttfamily node\mbox{[}1,2\mbox{]},node3}
\item Island=1 Nodes={\ttfamily node\mbox{[}1-\/3\mbox{]},node4}
\end{DoxyItemize}
\end{DoxyItemize}
\end{DoxyItemize}

Please visit the ./\+R\+E\+A\+D\+ME.islands.\+md \char`\"{}islands example\char`\"{} for more information and examples of a cluster configuration in form of islands. 