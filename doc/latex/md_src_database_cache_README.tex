The E\+AR Database Daemon (E\+A\+R\+D\+BD) caches the records generated by the ../library/\+R\+E\+A\+D\+ME.md \char`\"{}\+E\+A\+R\+L\char`\"{} and ../daemon/\+R\+E\+A\+D\+ME.md \char`\"{}\+E\+A\+R\+D\char`\"{} in the system and report it to the centralized database. Several E\+A\+R\+D\+BD daemons will run in the cluster reducing the number of connections and messages to the database and providing aggregated metrics of some of the records.

\subsection*{Requirements }

E\+A\+R\+D\+BD uses periodic power metrics sended by ../daemon/\+R\+E\+A\+D\+ME.md \char`\"{}\+E\+A\+R\+D\char`\"{}, the per-\/node daemon, including job identification details (job id, step id when executed in a slurm system).

\subsection*{Configuration }

The E\+AR Database Daemon uses the {\ttfamily /ear.conf} file to be configured. It can be dynamically configured by reloading the service.


\begin{DoxyCode}
1 # Fields related to the Database Daemon
\end{DoxyCode}


\subsection*{Execution }

To execute this component, this {\ttfamily systemctl} command examples are provided\+:
\begin{DoxyItemize}
\item {\ttfamily sudo systemctl start eard} to start the E\+A\+R\+D\+BD service.
\item {\ttfamily sudo systemctl stop eard} to stop the E\+A\+R\+D\+BD service.
\item {\ttfamily sudo systemctl reload eard} to force to reload the configuration of the E\+A\+R\+D\+BD service.
\end{DoxyItemize}

\subsection*{A\+PI }

The Database Daemon offers a simple A\+PI ({\ttfamily \hyperlink{eardbd__api_8c}{eardbd\+\_\+api.\+c}}) to be notified about a metric or event of a node. The A\+PI just sends this data, which will be cached and stored in the database by a buffering technique, reducing the cluster resource utilization. It uses sockets to transfer the data, allowing to the administrator to select both T\+CP or U\+DP protocols. U\+DP is not recommended because it can add too much mess to the E\+A\+R\+D\+BD synchronization.

\subsection*{License }

All the files in the E\+AR framework are under the L\+G\+P\+Lv2.\+1 license. See the \href{../../COPYING}{\tt C\+O\+P\+Y\+I\+NG} file in the E\+AR root directory. 